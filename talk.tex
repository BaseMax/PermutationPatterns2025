\documentclass[11pt]{article}
\usepackage[margin=1in]{geometry}
\usepackage{amsmath, amssymb}
\usepackage{graphicx}
\usepackage{titlesec}
\usepackage{enumitem}
\usepackage{lmodern}
\usepackage{hyperref}
\usepackage{xcolor}

\definecolor{titleblue}{RGB}{10,45,120}
\definecolor{graytext}{gray}{0.2}

\titleformat{\section}{\normalfont\Large\bfseries\color{titleblue}}{}{0pt}{}
\titleformat{\subsection}{\normalfont\large\bfseries\color{titleblue}}{}{0pt}{}

\renewcommand{\baselinestretch}{1.15}

\title{\textbf{Using SAT Solvers to Prove Pattern Containment}}
\author{Seyyed Ali Mohammadiyeh \\
	Department of Pure Mathematics, University of Kashan \\
	\texttt{max@std.kashanu.ac.ir}, \texttt{alim@kashanu.ac.ir}
	}

\date{}

\begin{document}
	
	\maketitle
	
	\vspace{-1.5em}
	\section*{Abstract}
	
	In this work, we explore the use of modern \textbf{SAT solvers} as a computational tool for investigating \textbf{permutation pattern containment and avoidance}. Specifically, we encode the existence of permutations containing or avoiding a given set of patterns as a Boolean satisfiability problem. This allows us to leverage the remarkable efficiency of SAT solvers to answer containment questions that are otherwise difficult to resolve analytically.
	
	\medskip
	
	We present a general-purpose encoding that maps permutations of length $n$ and target patterns of length $k$ into propositional logic, preserving the semantics of pattern containment. This encoding is optimized to reduce variable count and solver time, and supports both classical patterns and \textbf{vincular (adjacent)} constraints.
	
	\medskip
	
	Our implementation allows for automated generation of minimal examples, counterexamples, and can be used to confirm or refute \textbf{Wilf-equivalences}, even in cases where traditional enumeration or generating function methods fall short.
	
	\medskip
	
	As a case study, we analyze several open questions on pattern pairs of lengths 4 and 5, demonstrating how SAT-based techniques can provide computational evidence and insight, as well as full proofs in bounded-length domains.
	
	\medskip
	
	This approach opens up a promising new direction at the interface of \textbf{combinatorics, formal logic, and algorithm design}, and provides a reusable framework for future explorations in pattern avoidance and permutation class classification.
	
	\vfill
	\noindent
	\textbf{Keywords:} SAT solving, permutation patterns, pattern containment, pattern avoidance, Wilf-equivalence, combinatorics, logic encoding
	
\end{document}
